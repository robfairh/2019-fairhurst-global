\documentclass{anstrans}
%%%%%%%%%%%%%%%%%%%%%%%%%%%%%%%%%%%
\title{Implementation of Demand Driven Deployment Capabilities in \Cyclus for Transition Scenarios}
\author{Roberto E. Fairhurst$^a$, Gwendolyn J. Chee$^a$, Jin Whan Bae$^a$, Robert R. Flanagan$^b$, Kathryn D. Huff$^a$}

\institute{
$^a$University of Illinois at Urbana-Champaign, Dept. of Nuclear, Plasma, and Radiological Engineering\\
$^b$University of South Carolina, Nuclear Engineering Program\\
ref3@illinois.edu
}

%%%% packages and definitions (optional)
\usepackage{graphicx} % allows inclusion of graphics
\usepackage{booktabs} % nice rules (thick lines) for tables
\usepackage{microtype} % improves typography for PDF
\usepackage{xspace}
\usepackage{tabularx}
\usepackage{floatrow}
\usepackage{subcaption}
\usepackage{enumitem}
\usepackage{placeins}
\usepackage[acronym,toc]{glossaries}
\include{acros}

\newcommand{\SN}{S$_N$}
\renewcommand{\vec}[1]{\bm{#1}} %vector is bold italic
\newcommand{\vd}{\bm{\cdot}} % slightly bold vector dot
\newcommand{\grad}{\vec{\nabla}} % gradient
\newcommand{\ud}{\mathop{}\!\mathrm{d}} % upright derivative symbol
\newcommand{\Cyclus}{\textsc{Cyclus}\xspace}%
\newcommand{\Cycamore}{\textsc{Cycamore}\xspace}%
\newcolumntype{c}{>{\hsize=.56\hsize}X}
\newcolumntype{b}{>{\hsize=.7\hsize}X}
\newcolumntype{s}{>{\hsize=.74\hsize}X}
\newcolumntype{f}{>{\hsize=.1\hsize}X}
\newcolumntype{a}{>{\hsize=.45\hsize}X}
\usepackage{titlesec}
\titleformat*{\subsection}{\normalfont}

\begin{document}
%%%%%%%%%%%%%%%%%%%%%%%%%%%%%%%%%%%%%%%%%%%%%%%%%%%%%%%%%%%%%%%%%%%%%%%%%%%%%%%%
\section{Abstract}

Nuclear fuel cycle simulators require users to have prior knowledge regarding the fuel cycle under study.
Specifically, while most simulators automatically deploy reactor capacities to 
meet power demand, none predict corresponding fuel cycle facility deployments necessary to support 
those reactors. So, users must provide this information when defining the 
simulation. In practice, this requires that the user run the simulator several times until the implemented model reflects reality.
However, this sort of iteration can be simplified with a simulator capable of 
deploying all types of interdependent facilities automatically.
A new Demand Driven Deployment Capability --
\texttt{D3ploy} \cite{chee_demonstration_2019} -- 
is under development for the agent-based nuclear fuel cycle simulation framework \Cyclus \cite{huff_fundamental_2016}.
The objective of the present work was to carry out various simulations to prove 
\texttt{d3ploy}'s current capabilities for simulating complex cycles.

The Idaho National Laboratory Nuclear Fuel Cycle Evaluation and Screening Report \cite{wigeland_nuclear_2014} established
several fuel cycle scenarios.
As part of the project NEUP-FY16-10512, the simulations focused on the cases EG01, EG23, EG24, EG29, and EG30 (Table \ref{tab:1}).
All the scenarios started at EG01 -- representing the current U.S. fuel cycle -- and transitioned to advanced fuel cycles.
The simulations utilized \texttt{d3ploy}'s non-optimizing, deterministic-optimizing, and, currently under development,
stochastic-optimizing algorithms. The results obtained using the different algorithms emphasized \texttt{d3ploy}'s ability
to facilitate transition scenarios.

\floatsetup[table]{capposition=top}
\begin{table}[!htb]
    \begin{tabular}{|l|l|}
        \hline
        Fuel Cycle & Description                                                                                                                                 \\ \hline
        EG01       & \begin{tabular}[c]{@{}l@{}}Once-through using enriched-U fuel in \\ thermal critical reactors.\end{tabular}                                 \\ \hline
        EG23       & \begin{tabular}[c]{@{}l@{}}Continuous recycle of U/Pu with new\\ natural-U fuel in fast critical reactors.\end{tabular}                     \\ \hline
        EG24       & \begin{tabular}[c]{@{}l@{}}Continuous recycle of U/TRU with new\\ natural-U fuel in fast critical reactors.\end{tabular}                    \\ \hline
        EG29       & \begin{tabular}[c]{@{}l@{}}Continuous recycle of U/Pu with new\\ natural-U fuel in both fast and thermal\\ critical reactors.\end{tabular}  \\ \hline
        EG30       & \begin{tabular}[c]{@{}l@{}}Continuous recycle of U/TRU with new\\ natural-U fuel in both fast and thermal\\ critical reactors.\end{tabular} \\ \hline
    \end{tabular}
    \caption{Cases identified in \cite{wigeland_nuclear_2014}.}\label{tab:1}
\end{table}

%%%%%%%%%%%%%%%%%%%%%%%%%%%%%%%%%%%%%%%%%%%%%%%%%%%%%%%%%%%%%%%%%%%%%%%%%%%%%%%%

\section{Introduction}
%What is this section for?
%Here I can talk about the cases

\subsection{EG01}



\subsection{EG23}

\subsection{EG24}

\subsection{EG29}

\subsection{EG30}

\section{Method}
%How did I fill the gap?
%I can talk about the how I set up the transition scenarios without d3ploy first and then with it.

\section{Results}
%How should I present that?
%One plot for each case showing power??
%Showing some fuel??

\section{Conclusions}

\bibliographystyle{ans}
\bibliography{bibliography}
\end{document}

