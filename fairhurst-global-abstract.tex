\documentclass{anstrans}
%%%%%%%%%%%%%%%%%%%%%%%%%%%%%%%%%%%
\title{Implementation of Demand Driven Deployment Capabilities in \Cyclus for Transition Scenarios}
\author{Roberto E. Fairhurst$^a$, Gwendolyn J. Chee$^a$, Jin Whan Bae, Robert R. Flanagan$^b$, Kathryn D. Huff$^a$}

\institute{
$^a$University of Illinois at Urbana-Champaign, Dept. of Nuclear, Plasma, and Radiological Engineering\\
$^b$University of South Carolina, Nuclear Engineering Program\\
ref3@illinois.edu
}

%%%% packages and definitions (optional)
\usepackage{graphicx} % allows inclusion of graphics
\usepackage{booktabs} % nice rules (thick lines) for tables
\usepackage{microtype} % improves typography for PDF
\usepackage{xspace}
\usepackage{tabularx}
\usepackage{floatrow}
\usepackage{subcaption}
\usepackage{enumitem}
\usepackage{placeins}
\usepackage[acronym,toc]{glossaries}
\include{acros}

\makeglossaries
\newcommand{\SN}{S$_N$}
\renewcommand{\vec}[1]{\bm{#1}} %vector is bold italic
\newcommand{\vd}{\bm{\cdot}} % slightly bold vector dot
\newcommand{\grad}{\vec{\nabla}} % gradient
\newcommand{\ud}{\mathop{}\!\mathrm{d}} % upright derivative symbol
\newcommand{\Cyclus}{\textsc{Cyclus}\xspace}%
\newcommand{\Cycamore}{\textsc{Cycamore}\xspace}%
\newcolumntype{c}{>{\hsize=.56\hsize}X}
\newcolumntype{b}{>{\hsize=.7\hsize}X}
\newcolumntype{s}{>{\hsize=.74\hsize}X}
\newcolumntype{f}{>{\hsize=.1\hsize}X}
\newcolumntype{a}{>{\hsize=.45\hsize}X}
\usepackage{titlesec}
\titleformat*{\subsection}{\normalfont}

\begin{document}
%%%%%%%%%%%%%%%%%%%%%%%%%%%%%%%%%%%%%%%%%%%%%%%%%%%%%%%%%%%%%%%%%%%%%%%%%%%%%%%%
\section{Abstract}

Nuclear fuel cycle simulators require some previous knowledge on part of the users about the fuel cycle they want to study.
Sometimes the user has to carry out some calculations by hand in order to know how many facilities are necessary to deploy
to adequately represent such cycle. Consequently, it is usually required to run several times the simulator until the cycle
portrays correctly reality. However, this sort of iteration can be simplified by having a simulator capable of deploying the right
amount of facilities automatically. A new Demand Driven Deployment Capability (\texttt{d3ploy} \cite{chee_demonstration_2019})
is under development to make the agent-based nuclear fuel cycle simulation framework \Cyclus \cite{huff_fundamental_2016} have
those functionalities. The objective of the present work was to carry out several simulations making use of \texttt{d3ploy}.
The results showed that for some cases this capability was very helpful. Additionally, the results brought to light the need of
adding new features to \texttt{d3ploy} for more complex fuel cycles.
\\
As part of the project NEUP-FY16-10512, this work simulated several scenarios using \Cyclus. The Idaho National Laboratory
Nuclear Fuel Cycle Evaluation and Screening Report \cite{wigeland_nuclear_2014} established such scenarios. The simulations
focused on the cases EG01, EG23, EG24, EG29, and EG30, see Table \ref{tab:1}. All the scenarios started at EG01, representing
the current U.S. fuel cycle, and transition to advanced fuel cycles. Following, the simulations made use of the aforementioned
capability \texttt{d3ploy}. This capability counts with non-optimizing, deterministic-optimizing, and, currently under
development, stochastic-optimizing algorithms. The results obtained using the different algorithms emphasized the ability of
\texttt{d3ploy} to make easier the implementation of the transition scenarios.

\floatsetup[table]{capposition=top}
\begin{table}[!htb]
    \begin{tabular}{|l|l|}
        \hline
        Fuel Cycle & Description                                                                                                                                 \\ \hline
        EG01       & \begin{tabular}[c]{@{}l@{}}Once-through using enriched-U fuel in \\ thermal critical reactors.\end{tabular}                                 \\ \hline
        EG23       & \begin{tabular}[c]{@{}l@{}}Continuous recycle of U/Pu with new\\ natural-U fuel in fast critical reactors.\end{tabular}                     \\ \hline
        EG24       & \begin{tabular}[c]{@{}l@{}}Continuous recycle of U/TRU with new\\ natural-U fuel in fast critical reactors.\end{tabular}                    \\ \hline
        EG29       & \begin{tabular}[c]{@{}l@{}}Continuous recycle of U/Pu with new\\ natural-U fuel in both fast and thermal\\ critical reactors.\end{tabular}  \\ \hline
        EG30       & \begin{tabular}[c]{@{}l@{}}Continuous recycle of U/TRU with new\\ natural-U fuel in both fast and thermal\\ critical reactors.\end{tabular} \\ \hline
    \end{tabular}
    \caption{Cases identified in \cite{wigeland_nuclear_2014}.}\label{tab:1}
\end{table}

%%%%%%%%%%%%%%%%%%%%%%%%%%%%%%%%%%%%%%%%%%%%%%%%%%%%%%%%%%%%%%%%%%%%%%%%%%%%%%%%
\bibliographystyle{ans}
\bibliography{bibliography}
\end{document}

